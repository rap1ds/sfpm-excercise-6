\documentclass[a4paper]{article}

% Encoodaus, joka sopii suomenkielellä (esim. ä ja ö)
\usepackage[utf8]{inputenc}
\usepackage[T1]{fontenc}

% Suomenkielinen tavutus
\usepackage[finnish]{babel}

% Viitteet
\usepackage{natbib}

% Otsikkojen päätteetön fontti
\usepackage{sectsty}
\allsectionsfont{\sffamily\large}

% Viitteiden merkit
\bibpunct{(}{)}{;}{a}{,}{,}

% page counting, header/footer
\usepackage{fancyhdr}
\usepackage{lastpage}
\usepackage{ifthen}

\pagestyle{fancy}
\lhead{\footnotesize \parbox{11cm}{} }
\rhead{\footnotesize \parbox{11cm}{} }
\lfoot{\footnotesize \parbox{11cm}{Created with Dippa Editor \\ http://mkos.futupeople.com/dippa}}
\rfoot{\ifthenelse{\not \value{page}=\LastPagenum}{please turn over}{}}
% \cfoot{}
% \rhead{\footnotesize 3}
% \rfoot{\footnotesize Page \thepage\ of \pageref{LastPage}}
% \renewcommand{\headheight}{24pt}
% \renewcommand{\footrulewidth}{0.4pt}
\renewcommand{\headrulewidth}{0.0pt}

% \makeatletter
% \def\LastPagenum{\@ifundefined{r@LastPage}{0}{\expandafter
% \expandafter\expandafter\@cdr
% \csname r@LastPage\endcsname\@nil\null} }
% \makeatother
% \rfoot{\ifthenelse{\isodd{\value{page}} \and \not
% \value{page}=\LastPagenum}{please turn over}{}}

\begin{document}

\title{\small T-76.5612 Software Project Management \\ Exercise 7, Pre-Lecture \\ \huge Muutos ketterään kehitykseen}
\date{27.2.2012}
\author{Mikko Koski \\ mikko.koski@aalto.fi \\ 66467F}
\maketitle

\normalsize

\section{Johdanto}

Ericsson on ruotsalainen telekommunikaatiojärjestelmien valmistaja \citep{wikiericsson}. Telekommunikaatioala pohjautuu vahvasti standardeihin ja valtiollisiin määräyksiin, josta johtuen ala on perinteisesti ollut hyvin kankea ja hidasliikkeinen. Ericsson teki kuitenkin vuosina 2008-2012 radikaalin muutoksen, jossa organisaation vanhat ja kankeat työtavat korvattiin ketterillä menetelmillä. \citep{mikkonen2011}

Tämän esseen tarkoitusta on tarkastella Ericssonin kokemusten perusteella muutosta perinteisistä menetelmistä kohti ketteriä menetelmiä.

\section{Perinteisestä ketterään}

Muutos perinteisistä menetelmistä ketteriin menetelmiin isossa organisaatiossa on suuri askel tuntemattomaan. Suuren muutoksesta tekee se, että se ei ole pelkästään muutos organisaatiorakenteessa tai istumajärjestyksessä, vaan koko ajattelumaailma on muutettava kohti ketterää Lean-ajattelua. Vanhoista työtavoista pois oppiminen vaatii työtä sekä runsaasti rohkeutta. On siis selvää, että tämän laajuiset muutokset kohtaavat haasteista.

Mielestäni Ericsson onnistui muutoksessaan ison organisaation haasteista huolimatta todella hyvin. Yksi tärkeimmistä onnistumisen aineksista oli mielestäni se, että Ericsson ei alloittanut suoraan massiivista organisaatiomuutosta, vaan uusia menetelmiä kokeiltiin pienen testiryhmän avulla \citep{mikkonen2011}. Pienen testiryhmän avulla organisaatio saa kerättyä itselleen tärkeää tietoa siitä mikä toimii ja mikä ei. Testiryhmän positiiviset kokemukset saattavat myös vakuuttaa organisaation muut työntekijät mallin toimivuudesta ja näin helpottaa malliin siirtymistä.

Ericssonilla ymmärrettiin mielestäni hyvin muutoksen vaatima perinpohjaisuus. Ketterien menetelmien käyttöönotto vaikuttaa laajalla skaalalla kaikkeen, mitä yhtiö tekee. Se vaikuttaa niin ajattelutapaan kuin työkaluihinkin. Luulen, että Ericsson ei olisi onnistunut muutoksessaan näin hyvin, jos työskentelyssä olisi muutettu vain prosessimalli.

Sen sijaan Ericssonilla toteutettiin siirtyminen ketteriin menetelmiin laajalla rintamalla. Prosessien lisäksi organisaatio muutti koko johtamistyylien antaen työntekijöilleen enemmän vaikuttamismahdollisuutta vähentäen ylhäältä-alas kontrollointia. Samalla tiimien, henkilökohtaisen kehittymisen, rohkeuden ja kommunikaation merkitystä korostettiin. Tämä oli mielestäni yksi onnistumisen edellytyksistä, sillä Scrum ei toimi ympäristössä, jossa tiimin jäseniä käsketään ylhäältä-alas.

\section{Muutoksen haasteet}

Muutos organisaation työtavoissa saatetaan nähdä vaikeana toteuttaa ja mielestäni se on hyvin ymmärrettävää. Ericssonin tapauksessa oli muutamia hyviä esimerkkejä esteistä, joita suuren luokan muutos kohtaa.

Ericssonin tapauksessa yritys joutui jatkuvasti perustelemaan itselleen tarvetta muutokselle. Yhtiö ei kohdannut suuria haasteista, jotka pakottivat muutokseen vaan päinvastoin, yhtiöllä meni hyvin ennen muutosta. \citep{mikkonen2011}

Ennen muutosta yhtiön oli vaikea nähdä hyötyjä, joita muutos mukanaan voisi tuoda. Loppujen lopuksi yhtiön piti vain luottaa positiivisiin kokemuksiin muilta yhtiöiltä sekä omaan intuitioon.

On selvää, että tällaiset ennakkolähtökohdat eivät ole suotuisat isolle muutokselle. Kun hyötyjä ei nähdä ennen muutosta on muutoksen perustelu organisaatiolle varmasti hankalaa. Mielestäni Ericsson kuitenkin toimi asiassa oikein, eli kokeili uutta mallia pienellä ryhmällä ennen suuren muutoksen aloittamista. Rohkeudesta, jota Ericssonilla oli aloittaessaa muutosta on myös nostettava hattua.

\section{Mistä haluaisin tietää lisää?}

Artikkeleissa, joissa kerrotaan Ericssonin muutoksesta \citep{mikkonen2011} \citep{hallikainen2012} kuvaillaan pääosin muutoksen onnistumisesta eikä juurikaan mainita, että muutoksesta olisi seurannut juuri mitään huonoa. Artikkeleissa ei myöskään puhuta kovinkaan paljon ongelmista, joita organisaatio muutoksen aikana kohtasi. Olisi mielenkiintoista tietää enemmän todellisista ja kivuliaistakin ongelmista, joita muutos aiheutti. Miten näistä ongelmista selvittiin? Artikkelissa annetaan esimerkki vastarinnasta uutta istumajärjestystä kohtaan, joka kuitattiin henkilökohtaisilla keskusteluilla. Uskaltanen epäillä, että kaikkein vaikeimpia ongelmia ei varmastikaan pelkällä juttelulla selvitetä.

Olisi myös mielenkiintoista tietää, miten organisaatio ratkaisi ongelmat isojen hajautettujen projektien kommunikoinnissa. \citet{mikkonen2011} toteaa artikkelissaan, että kommunikaatiossa ei voida luottaa pelkästään Scrum tiimien Scrumiin, muttei tarjoa ongelmaan konkreettista ratkaisua. Uskaltanen tässäkin väittää, että vaikeimpia kommunikaatio-ongelmia suurissa hajautetuissa projekteissa ei ratkaista pelkästään konferenssivideolla, Wiki-ympäristöllä ja chat-huoneilla, kuten \citet{hallikainen2012} antaa artikkelissaan ymmärtää.

Viimeisenä aiheena haluaisin tietää kokemuksista Kanbanin käytöstä. \citet{seikola} kuvaa artikkelissaan Kanbanin käytöstä ylläpidon apuna, mutta olisin kiinnostunut kuulemaan millaisia kokemuksia Kanbanista on tuotekehityskäytössä.

\section{Yhteenveto}

Ericsson toteutti muutoksen ketteriin menetelmiin mielestäni todella hienosti. Artikkelien perusteella voin todeta, että olisin itse toiminut varmasti samoin kuin Ericsson. Artikkelit eivät myöskään juurikaan tuoneet esille ongelmia tai epäonnistumisia muutosprosessissa, joten tehtävänannossa pyydettyä kriittistä näkökulmaa esseeseen oli hieman hankala näiden artikkelien pohjalta löytää.

% 1. Based on the articles discuss critically the agile transformation at Ericsson.
% 2. Choose 1-3 topics that you would you like to learn more from Ericsson about the transformation their working practices etc. Explain what you would like to hear more and why that might be interesting.

\bibliographystyle{plainnat}
\bibliography{ref}

\end{document}